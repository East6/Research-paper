\chapter{今後の課題}

この章では,鍵方式の実装・検証について,今後の課題を列挙し,説明をする.\\
以下に,今後の課題を列挙する.
\begin{itemize}
    \item セキュリティの脆弱性の改善
    \begin{itemize}
        \item http通信を行なっていることの改善
        \item 秘密鍵を通信していることの改善
        \item CSRFトークンOFF\cite{CSRF-token}した上での実装
    \end{itemize}
    \item 鍵方式の登録・認証がセキュアな上での検証を行う
    \item 利用者視点で使いやすいようにする
        \begin{itemize} 
            \item 鍵を紛失した時の再発行をできるようにする
            \item 鍵の管理を行いやすいようにする
            \item 鍵方式を知らない人でも使えるようにようにする
    \end{itemize}

\end{itemize}
上記に列挙した今後の課題の説明を以下に記述する.

「セキュリティの脆弱性の改善」と「鍵方式の登録・認証がセキュアな上での検証を行う」については,
現実世界に反映することができるようにするために,改善していく必要がある.
「利用者視点で使いやすいようにする」については,
%検証2での検証,考察で次のようなこと考えられるので,改善したい.
%筆者が構築した鍵方式は,利用者目線だと,使いづらい認証方式になっている.
検証2での検証,考察で"筆者が構築した鍵方式は,利用者目線だと,使いづらい認証方式"
ということが考えられるので改善する必要がある.
