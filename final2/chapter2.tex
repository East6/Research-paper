\chapter{基礎概念}
\label{chap:concept}

\section{認証技術}
「認証」とは、人やモノ、情報の正当性を証明することであり,人やモノを対象に した認証を「主体認証」\cite{technology-authentication}という.
主体認証には,「主体の知識による認証」「主体の所有する者による認証」「主体の身体的な特性による認証」の3種類がある.
主体の知識による認証による代表例は,パスワード認証,パスフレーズ認証,および,PIN(Personal Identification Number)を用いたものである.これらは実装が安易な認証技術であり,多くのITシステムで採用されている\cite{technology-authentication-2}.
所有するものによる認証では,携行可能な物理的デバイスを利用する.物理的デバイスとしては,ICカードや,ワンタイムパスワードでも利用されるようなハードウェアトークンなどがある\cite{technology-authentication-2}.
身体的な特性による認証には,利用者の指紋,色彩や生脈パターンなどを用いたものがある.これらをまとめて,バイオメトリクス認証(生体認証)と呼ぶ\cite{technology-authentication-2}.

\section{公開鍵暗号方式}
公開鍵暗号方式は,暗号化鍵と複合鍵を別のものとするRSAのような暗号化方式\cite{technology-SSH_authentication}である. 
秘密鍵から公開鍵を求めることはできるが,公開鍵から秘密鍵を求めることは困難\cite{technology-SSH_authentication}である.

\section{公開鍵暗号方式を用いたSSH認証}%参考文献を追加したい!!!!!!!!!!!!!
秘密鍵はクライアントが持ち,公開鍵はサーバ側が持つ.
公開鍵認証によるSSHの流れを次に記述する.
\textcircled{\scriptsize 1}クライアントがサーバにssh認証をする.
\textcircled{\scriptsize 2}サーバは公開鍵を用いて,ランダムな値の暗号化を行い,クライアントに送信する.
\textcircled{\scriptsize 3}クライアントは,送信された暗号化の値を解読し,サーバに送る.
\textcircled{\scriptsize 4}サーバは,\textcircled{\scriptsize 2}で行った暗号化される前の値と,\textcircled{\scriptsize 3}で送られてきた,解読された値が一致しているか確認する.一致していたら認証成功にし,不一致なら認証失敗にする.



\section{提案手法の実装で用いた技術}
    \subsection{HyperText Transfer Protocol(HTTP)}
        HTTPについての説明 hogehoge
        \subsubsection{GETメソッド}
        \subsubsection{POSTメソッド}
        % ページが足りなかったら,deleteメソッドとか入れるのあり
    \subsection{Ruby on Rails}
    \subsection{node.js}
        node.jsの説明 hogehoge
        \subsubsection{ssh2(モジュール)}% githubのreadmeのところを参考にしよう!
        \subsubsection{socket.io(モジュール)}% githubのreadmeのところを参考にしよう!
    \subsection{javascript}
    \subsection{ssh(open-ssh?)}
    \subsection{websocket}
    \subsection{cookie}
    \subsection{セッション}
        \subsubsection{ステートフル????}
        \subsubsection{ステートレス????}
        \subsubsection{一時セッション}% railsチュートリアルから
        \subsubsection{永続セッション}% railsチュートリアルから
% ページ数足りなかったら
%% html
%% css