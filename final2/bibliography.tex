% 参考文献
\def\line{−\hspace*{-.7zw}−}

\begin{thebibliography}{99}
%\bibitem{*}内の * は各自わかりやすい名前などをつけて、
%論文中には \cite{*} のように使用する。
%これをベースに書き換えた方が楽かも。
%書籍、論文、URLによって若干書き方が異なる。
%URLを載せる人は参考にした年月日を最後に記入すること。


%\bibitem{hoge}
%hoge

% 技術概要のところ
\bibitem{technology-authentication} \url{https://www.fom.fujitsu.com/goods/pdf/security/fpt1610-2.pdf} \\最終閲覧日:2019/10/10
\bibitem{technology-authentication-2} マスタリングTCP/IP 情報セキュリティ編 第1版 , 2014 , オーム社 , p68 〜 p75 
\bibitem{technology-SSH_authentication} マスタリングTCP/IP 情報セキュリティ編 第1版 , 2014 , オーム社 , p34 

% 提案手法ボツ案? のところだったはず!
\bibitem{cookie1} \url{https://tools.ietf.org/html/rfc6265#section-1} \\ 最終閲覧日:2020/1/20
\bibitem{cookie2} \url{https://qiita.com/mogulla3/items/189c99c87a0fc827520e} \\ 最終閲覧日:2020/1/20
\bibitem{cookie-httponly-default} \url{https://qiita.com/yasu/items/8ae3077bdbee606681f6#cookiestore%E3%81%8C%E5%95%8F%E9%A1%8C%E3%81%AA%E3%81%AE%E3%81%8B} \\ 最終閲覧日:2020/1/21
\bibitem{cookie-httponly-security} \url{https://developer.mozilla.org/ja/docs/Web/HTTP/Cookies} \\最終閲覧日:2020/1/21
\end{thebibliography}
