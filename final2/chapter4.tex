\chapter{検証}
\label{chap:poordirection}


\section{検証1}

\subsection{検証背景}

\subsubsection{検証目的}

password認証方式の,新規登録・認証の面倒さを解決するために,第3章の提案手法で
password認証方式に変わる,公開鍵暗号方式によるssh認証を用いた,WEBサービス認証の提案を行った。\\
しかしながら,提案だけだと,新規登録・認証の面倒さを解決していることの根拠に乏しい。
よって,検証を行い,第3章の提案手法は"面倒さの軽減"に効果的に繋がっているかの確認をする。


\subsubsection{検証手段}

検証の手段としては,実際に,以下の2っの認証方式の登録・認証を被験者に体験してもらう。
%また,実験マニュアルを作成し,実験者は被験者に対して,同じように接するようにする。

\begin{itemize}
  \item パスワード方式認証(以後 "パスワード認証"と記述する)
  \item 公開鍵暗号暗号方式によるssh認証(以後 "鍵認証”と記述する)
\end{itemize}
その後,2っの観点から,"面倒さ"を数値化する。

1っ目の観点は「時間」である。
"面倒さ"をアカウント登録・認証にかかる時間と推測し,計測化する。
詳しい詳細については,検証環境のマニュアルに記述する。

%まず,パスワード 形式による登録・認証の時間計測をそれぞれ行う。
% 次に,公開鍵暗号方式によるSSH 形式登録・認証の時間計測をそれぞれ行う。
% 最後に,上記の形式による時間計測を比較する。

2っ目の観点は「アンケート」である。
アンケートには,点数で答える方式,文字で記入する欄 の2っがあり,点数で答える方式により"面倒さ”を数値化する。
アンケートの細かい内容は,4.1.3の検証画面に記述する。

また,アンケートには"面倒さ"を数値化する以外にも,以下の2っの意味を込める。
1っ目の意味は次の通りである。
アカウント登録・認証にかかる時間 を,"面倒さ"と予想して検証しているが,その予想を確かめる必要がある。
アンケートを取ることにより,「アカウント登録・認証にかかる時間」と,「アンケートによる面倒さ」が比例していることを確認することで,予想を確かめることができる。
また,被験者の状態も確認することで,面倒と感じるのが,検証自体に対しての面倒さと関係があるのかを確認する。
2っ目の意味は次のとおりである。
記入欄で,改善点や感じたことの意見をもらうことで,今後の研究に生きるようなアンケートをもらう。








\subsection{検証環境}
 \subsubsection{検証場所}
 第3章で記述したとおり,検証場所は学科のVMを用いているため,学科のネットワーク内(有線LAN,wifiアクセスポイント{ie-ryukyu})
 から,アクセスして検証を行う。


 %\subsubsection{検証画面}

 \subsubsection{検証の流れ}
    ここでは,被験者に行ってもらう,検証の流れを記述する。
    時間の観点で"面倒さ"を数値化する検証では,被験者にはパスワード方式,鍵方式の登録・認証をそれぞれ行ってもらう。その時,被験者は時間を測る。 
    また,再現性を持って,検証を行うためにマニュアルを作成し,マニュアル通りに検証を行う。
    アンケートの観点で"面倒さ"を数値化する検証では,
    時間の観点で"面倒さ"を数値化する検証 が終わった直後に行うようにすることで,被験者の思った感情とアンケート結果の差異が少なくなるようにする。
  

  \subsubsection{検証画面}
    %実際に記述--------------------------------------------------------
    ここでは,被験者に行ったもらう検証画面をのせる。

    以下の
    図\ref{検証1アカウント作成(パスワード方式)},
    図\ref{検証1認証(パスワード方式)}
    図\ref{検証1認証成功後(パスワード方式)}
    は,第3章の提案手法で実現したパスワード方式,登録・認証を,実際に被験者に行ってもらった時のブラウザ画面である。
    図\ref{検証1アカウント作成(パスワード方式)}は,アカウント登録画面である。
    図\ref{検証1認証(パスワード方式)}は,図\ref{検証1アカウント作成(パスワード方式)}で登録したアカウントに認証するための画面である。
    図\ref{検証1認証成功後(パスワード方式)}は,図\ref{検証1認証(パスワード方式)}で認証成功した後の画面である。
    % パスワード方式のスクショ---------------------------
    \vspace{4cm}%図の位置を正しくする!
    %\begin{figure}[h]
    \begin{figure}[H]
        %\centering
        \includegraphics[width=15cm]{./fig/chapter4/inspect_1/password_screnn/sign_up.png}
        \caption{検証1\_アカウント作成(パスワード方式)}
        \label{検証1アカウント作成(パスワード方式)}
    \end{figure}

    \vspace{4cm}%図の位置を正しくする!
    \begin{figure}[H]
        %\centering
        \includegraphics[width=15cm]{./fig/chapter4/inspect_1/password_screnn/login.png}
        \caption{検証1\_認証(パスワード方式)}
        \label{検証1認証(パスワード方式)}
    \end{figure}

    \vspace{4cm}%図の位置を正しくする!
    \begin{figure}[H]
        %\centering
        \includegraphics[width=15cm]{./fig/chapter4/inspect_1/password_screnn/success.png}
        \caption{検証1\_認証成功後(パスワード方式)}
        \label{検証1認証成功後(パスワード方式)}
    \end{figure}
    % パスワード方式のスクショ---------------------------








    \newpage
    \newpage

    以下の
    図\ref{検証1アカウント作成(鍵方式)},
    図\ref{検証1認証(鍵方式)}
    図\ref{検証1認証成功後(鍵方式)}
    は,第3章の提案手法で実現した,鍵方式の登録・認証を,実際に被験者に行ってもらった時のブラウザ画面である。
    図\ref{検証1アカウント作成(鍵方式)}は,アカウント登録画面である。
    図\ref{検証1認証(鍵方式)}は,図\ref{検証1アカウント作成(鍵方式)}で登録したアカウントに認証するための画面である。
    図\ref{検証1認証成功後(鍵方式)}は,図\ref{検証1認証(鍵方式)}で認証成功した後の画面である。
    % 鍵方式のスクショ----------------------------------
    \vspace{4cm}%図の位置を正しくする!
    \begin{figure}[H]
        %\centering
        \includegraphics[width=15cm]{./fig/chapter4/inspect_1/key_screnn/sign_up.png}
        \caption{検証1\_アカウント作成(鍵方式)}
        \label{検証1アカウント作成(鍵方式)}
    \end{figure}

    \vspace{4cm}%図の位置を正しくする!
    \begin{figure}[H]
        %\centering
        \includegraphics[width=15cm]{./fig/chapter4/inspect_1/key_screnn/login.png}
        \caption{検証1\_認証(鍵方式)}
        \label{検証1認証(鍵方式)}
    \end{figure}

    \vspace{4cm}%図の位置を正しくする!
    \begin{figure}[H]
        %\centering
        \includegraphics[width=15cm]{./fig/chapter4/inspect_1/key_screnn/login-after2.png}
        \caption{検証1\_認証成功後(鍵方式)}
        \label{検証1認証成功後(鍵方式)}
    \end{figure}
    % 鍵方式のスクショ----------------------------------





    

    以下の
    図\ref{アンケート1},
    図\ref{アンケート2},
    図\ref{アンケート3},
    図\ref{アンケート4}
    は
    図\ref{検証1アカウント作成(パスワード方式)} 〜 図\ref{検証1認証成功後(鍵方式)} までの,
    時間計測が終わった直後に行う,アンケートの画面である。
    %% アンケートの意図を説明
    % 検証自体面倒か
    図\ref{アンケート1} では,"検証自体の面倒さ"をアンケートで聞いている。
    その意図として,"面倒"と感じた感情は,検証役になること自体が面倒だったことに起因していないかを確かめるためである。
    % 登録・認証に対して,面倒と感じた度合いを
    図\ref{アンケート2}では,パスワード方式の登録・認証それぞれについて,どのくらい面倒かを質問している。
    図\ref{アンケート3}では,鍵方式の登録・認証それぞれについて,どのくらい面倒かを質問している。
    図\ref{アンケート2},図\ref{アンケート3}の質問の意図としては,アンケートの観点から"面倒さ"を数値化することである。
    図\ref{アンケート4}では,鍵方式の登録・認証について,被験者の技術視点と利用視点から,自由形式で意見を求めている。
    図\ref{アンケート4}の意図としては,よりよい物を作るために,被験者意見を収集し,今後の方針を決めるためである。

    % アンケートのスクショ----------------------------------
    \vspace{4cm}%図の位置を正しくする!
    \begin{figure}[H]
        %\centering
        \includegraphics[width=15cm]{./fig/chapter4/inspect_1/questionnaire/questionnaire_1.png}
        \caption{アンケート1}
        \label{アンケート1}
    \end{figure}

    \vspace{4cm}%図の位置を正しくする!
    \begin{figure}[H]
        %\centering
        \includegraphics[width=15cm]{./fig/chapter4/inspect_1/questionnaire/questionnaire_2.png}
        \caption{アンケート2}
        \label{アンケート2}
    \end{figure}

    \vspace{4cm}%図の位置を正しくする!
    \begin{figure}[H]
        %\centering
        \includegraphics[width=15cm]{./fig/chapter4/inspect_1/questionnaire/questionnaire_3.png}
        \caption{アンケート3}
        \label{アンケート3}
    \end{figure}

    \vspace{4cm}%図の位置を正しくする!
    \begin{figure}[H]
        %\centering
        \includegraphics[width=15cm]{./fig/chapter4/inspect_1/questionnaire/questionnaire_4.png}
        \caption{アンケート4}
        \label{アンケート4}
    \end{figure}

    
    % アンケートのスクショ----------------------------------

    %実際に記述--------------------------------------------------------
    




    %%%%%%%%%%%%%%%%%%%%%%%%%%%アン
    %目的
    %  ここでは,被験者に行ったもらう検証画面をのせる。
    %  %(なぜか --> 
    %  % - 検証結果が,どういう検証をした上での,結果になったかを見る必要がある。
    %  % - プラスアルファ
    %  %   - 検証結果を受けて,変更したから,どういう変更したかを,被験者目線でわかるようにする
    %  %)
    %  
    %検証の詳細
    %  パスワード方式の認証・登録の検証画面
    %  鍵方式の認証・登録の検証画面
    %  アンケートの画面
    %
    %  
    %被験者に検証してもらう,鍵方式とパスワード方式それぞれの,登録・認証をする上での,図を載せる。
    %
    %%%%%%%%%%%%%%%%%%%%%%%%%%%アン

 \subsubsection{検証マニュアル}
 %再現性のためにという説明をする
 以下の図\ref{検証マニュアル1},図\ref{検証マニュアル2}は,上記の図\ref{検証1アカウント作成(パスワード方式)} 〜 図\ref{アンケート4} 
 の検証を行うためのマニュアルである.
 マニュアルを作成して,検証を行った意図としては,再現性を持って検証を行うためである.

 \newpage

 \vspace{4cm}%図の位置を正しくする!
 \begin{figure}[H]
     %\centering
     \includegraphics[width=15cm]{./fig/chapter4/inspect_1/manual/manual_1.pdf}
     \caption{検証マニュアル1}
     \label{検証マニュアル1}
 \end{figure}

 \vspace{4cm}%図の位置を正しくする!
 \begin{figure}[H]
     %\centering
     \includegraphics[width=15cm]{./fig/chapter4/inspect_1/manual/manual_2.pdf}
     \caption{検証マニュアル2}
     \label{検証マニュアル2}
 \end{figure}

 \subsection{検証結果}
 \subsection{考察}





\newpage





%% 検証2でやるやつ
%\section{検証2}
%  \subsection{検証背景}
%    \subsubsection{検証目的}
%    \subsubsection{検証手段} 
%  \subsection{検証環境}                        OK
%    \subsubsection{検証場所}                   OK
%    \subsubsection{検証の流れ}                 OK
%    \subsubsection{検証画面}                   OK
%    \subsubsection{検証マニュアル}              OK
%  \subsection{検証結果}
%  \subsection{考察}




\section{検証2}
  \subsection{検証背景}
    \subsubsection{検証目的}
    \subsubsection{検証手段} 
  \subsection{検証環境} 
    \subsubsection{検証場所}% 基本一緒?
       第3章で記述したとおり,検証場所は学科のVMを用いているため,学科のネットワーク内(有線LAN,wifiアクセスポイント{ie-ryukyu})
        から,アクセスして検証を行う。
    \subsubsection{検証の流れ}% 基本一緒?
        ここでは,被験者に行ってもらう,検証の流れを記述する。
        時間の観点で"面倒さ"を数値化する検証では,被験者にはパスワード方式,鍵方式の登録・認証をそれぞれ行ってもらう。その時,被験者は時間を測る。 
        また,再現性を持って,検証を行うためにマニュアルを作成し,マニュアル通りに検証を行う。
        アンケートの観点で"面倒さ"を数値化する検証では,
        時間の観点で"面倒さ"を数値化する検証 が終わった直後に行うようにすることで,被験者の思った感情とアンケート結果の差異が少なくなるようにする。


    \subsubsection{検証画面}

    ここでは,検証1より効果的に検証結果目的を達成するために,修正した検証2の検証画面とその意図について述べる。

    %ここで,検証を行った際に,検証1と比較して被験者が視覚的・聴覚的に差異が生じる主な変更点を以下に述べる。

    まずは検証画面図の説明をする。\\
    パスワード方式について
    \begin{itemize}
        \item 図\ref{検証2ホーム画面(パスワード方式)}は,ホーム画面である.
        \item 図\ref{検証2アカウント作成(パスワード方式)}は,アカウント登録画面である。
        \item 図\ref{検証2認証(パスワード方式)}は,図\ref{検証2アカウント作成(パスワード方式)}で登録したアカウントに認証するための画面である。
        \item 図\ref{検証2認証成功後(パスワード方式)}は,図\ref{検証2認証(パスワード方式)}で認証成功した後の画面である。
    \end{itemize}
    %
    鍵方式について\\
    \begin{itemize}
        \item 図\ref{検証2ホーム画面(鍵方式)}は,ホーム画面である.
        \item 図\ref{検証2アカウント作成(鍵方式)}は,アカウント登録画面である。
        \item 図\ref{検証2認証(鍵方式)}は,図\ref{検証2アカウント作成(鍵方式)}で登録したアカウントに認証するための画面である。
        \item 図\ref{検証2認証成功後(鍵方式)}は,図\ref{検証2認証(鍵方式)}で認証成功した後の画面である。
    \end{itemize}



    % 言いたいやつ----------------------------------------------------------------------------------------------

    %% web画面面 -----------------------------------------------------
    %%% アンケートの"検証自体面倒"という観点から
    %%%%検証1のアンケートで,"検証自体が面倒"と答える被験者の平均が,5段階中の1であった。その原因として,
    %%%%デバッグ画面や,登録フォーム・認証フォームだけのWEB画面が,
    %%%%作業的な検証に繋がり,”検証自体が面倒"に繋がった大きな要因の一つと考えられる。
    %%%%よって,ホーム画面(図???)の用意や,デバッグの消去(図???),さらには,「幸せになれるWEBサイト」と認識してもらうことで,被験者にとって,作業的な検証から,WEBサービスに登録する検証を狙って,変更した。
    %% web画面面 -----------------------------------------------------

    %% 細かいやつ
    %%%アンケートは検証1と同じ質問(意図)をしているため,割愛する。

    % 言いたいやつ----------------------------------------------------------------------------------------------

    次に,検証1 との変更点と,その意図について述べる。
    %アンケートの"検証自体面倒"という観点から
    検証1のアンケート(図\ref{検証2アカウント作成(鍵方式)})で,"検証自体が面倒"と答える被験者の平均が,5段階中の1であった。その原因として,
    デバッグ画面や,登録フォーム・認証フォームだけのWEB画面が,
    作業的な検証に繋がり,"検証自体が面倒"に繋がった大きな要因の一つと考えられる。
    よって,ホーム画面(図\ref{検証2ホーム画面(パスワード方式)},図\ref{検証2ホーム画面(鍵方式)})の用意や,
    デバッグのための文字の消去(図\ref{検証2アカウント作成(パスワード方式)} 〜 図\ref{検証2認証成功後(パスワード方式)} , \ref{検証2アカウント作成(鍵方式)} 〜 \ref{検証2アカウント作成(鍵方式)}),
    さらには,「幸せになれるWEBサイト」( 図\ref{検証2ホーム画面(パスワード方式)} 〜 \ref{検証2アカウント作成(鍵方式)} )と認識してもらうことで,
    被験者にとって,作業的な検証から,WEBサービスに登録する検証になることを狙って,変更した。









    % パスワード方式のスクショ---------------------------
    \vspace{4cm}%図の位置を正しくする!
    %\begin{figure}[h]
    \begin{figure}[H]
        %\centering
        \includegraphics[width=15cm]{./fig/chapter4/inspect_2/password_screnn/home.png}
        \caption{検証2\_ホーム画面(パスワード方式)}
        \label{検証2ホーム画面(パスワード方式)}
    \end{figure}


    \vspace{4cm}%図の位置を正しくする!
    %\begin{figure}[h]
    \begin{figure}[H]
        %\centering
        \includegraphics[width=15cm]{./fig/chapter4/inspect_2/password_screnn/sign_up.png}
        \caption{検証2\_アカウント作成(パスワード方式)}
        \label{検証2アカウント作成(パスワード方式)}
    \end{figure}

    \vspace{4cm}%図の位置を正しくする!
    %\begin{figure}[h]
    \begin{figure}[H]
        %\centering
        \includegraphics[width=15cm]{./fig/chapter4/inspect_2/password_screnn/login.png}
        \caption{検証2\_認証(パスワード方式)}
        \label{検証2認証(パスワード方式)}
    \end{figure}

    \vspace{4cm}%図の位置を正しくする!
    %\begin{figure}[h]
    \begin{figure}[H]
        %\centering
        \includegraphics[width=15cm]{./fig/chapter4/inspect_2/password_screnn/success.png}
        \caption{検証2\_認証成功後(パスワード方式)}
        \label{検証2認証成功後(パスワード方式)}
    \end{figure}
    % パスワード方式のスクショ---------------------------

    % 鍵方式のスクショ----------------------------------
    \vspace{4cm}%図の位置を正しくする!
    \begin{figure}[H]
        %\centering
        \includegraphics[width=15cm]{./fig/chapter4/inspect_2/key_screnn/home.png}
        \caption{検証2\_ホーム画面(鍵方式)}
        \label{検証2ホーム画面(鍵方式)}
    \end{figure}

    \vspace{4cm}%図の位置を正しくする!
    \begin{figure}[H]
        %\centering
        \includegraphics[width=15cm]{./fig/chapter4/inspect_2/key_screnn/sign_up.png}
        \caption{検証2\_アカウント作成(鍵方式)}
        \label{検証2アカウント作成(鍵方式)}
    \end{figure}

    \vspace{4cm}%図の位置を正しくする!
    \begin{figure}[H]
        %\centering
        \includegraphics[width=15cm]{./fig/chapter4/inspect_2/key_screnn/login.png}
        \caption{検証2\_認証(鍵方式)}
        \label{検証2認証(鍵方式)}
    \end{figure}

    \vspace{4cm}%図の位置を正しくする!
    \begin{figure}[H]
        %\centering
        \includegraphics[width=15cm]{./fig/chapter4/inspect_2/key_screnn/login-after2.png}
        \caption{検証2\_認証成功後(鍵方式)}
        \label{検証2認証成功後(鍵方式)}
    \end{figure}
    % 鍵方式のスクショ---------------------------------- 

    % アンケートは検証1と同じ質問(意図)をしているため,割愛する。








   
    \newpage


    \subsubsection{検証マニュアル}
    以下の図\ref{検証2マニュアル1},図\ref{検証2マニュアル2},図\ref{検証2マニュアル3}は,上記の図\ref{検証2アカウント作成(パスワード方式)} 〜 図\ref{検証2アカウント作成(鍵方式)} ,図\ref{アンケート1} 〜 図\ref{アンケート4}
    の検証を行うためのマニュアルである.
    マニュアルを作成して,検証を行った意図としては,再現性を持って検証を行うためである.

    次に,検証1のマニュアルと比べての相違点とその意図を以下に述べる.\\
    パスワードの使い回し,password というパスワードなど,日常的に設定しないと思われるパスワードの設定が見受けられた。
    そのため,マニュアル(図\ref{検証2マニュアル1})に 「普段用いるように」,と被験者に注意を促す。
    また,パスワードの設定に関して,日常的に設定ような,より効果的な検証を行うためには,
    アカウントの登録で,パスワードに対して,バリデーション(パスワードの制限)をすることがあげられる。
    しかし,今回の検証では,あえてバリデーションをかけないことにする。
    理由は,
        鍵認証方式の,セキュリティは脆弱性が大きいのに対して,
        パスワード方式のだけセキュアにすると,検証比較として適していないと判断したためである。

    %何をやっているのかわからない
    鍵方式での,登録・認証に関して「何やっているかわからない」という意見があった。
    よって,マニュアル(図\ref{検証2マニュアル1},図\ref{検証2マニュアル2})に,「"パスワード方式です" "鍵方式です"」と軽く説明を加える。
    また,鍵方式の登録・認証に対しての,説明を詳しくすると,
    一般的に普及している,パスワード方式 による,被験者の慣れ を検証に含めることができないので,
    軽く「"パスワード方式です" "鍵方式です"」という説明を加えることにする。

    % 言いたいやつ----------------------------------------------------------------------------------------------
    %% マニュアル面(被験者生の動作や生の声から)----------------------------

    %%% より,日常的な検証結果を得るために(パスワード設定)
    %%%%パスワードの使い回し,password というパスワードなど,日常的に設定しないと思われるパスワードの設定が見受けられた。
    %%%%そのため,マニュアル(図????)に 「普段用いるように」,と被験者に注意を促す。
    %%%%また,パスワードの設定に関して,日常的に設定ような,より効果的な検証を行うためには,
    %%%%アカウントの登録で,パスワードに対して,バリデーション(パスワードの制限)をすることがあげられる。
    %%%%しかし,今回の検証では,あえてバリデーションをかけないことにする。
    %%%%理由は,
    %%%%    鍵認証方式の,セキュリティは脆弱性が大きいのに対して,
    %%%%    パスワード方式のだけセキュアにすると,検証比較として適していないと判断したためである。

    %%% 何をやっているのかわからない
    %%%% 鍵方式での,登録・認証に関して「何やっているかわからない」という意見があった。
    %%%% よって,マニュアル(図????)に,「"パスワード方式です" "鍵方式です"」と軽く説明を加える。
    %%%% 鍵方式の登録・認証に対しての,説明を詳しくすると,
    %%%%    一般的に普及している,パスワード方式 による,被験者の慣れ を検証に含めることができないので,
    %%%% 軽く「"パスワード方式です" "鍵方式です"」という説明を加える。
    %% マニュアル面(被験者生の動作や生の声から)----------------------------

    % 言いたいやつ----------------------------------------------------------------------------------------------


    \newpage
    \vspace{4cm}%図の位置を正しくする!
    \begin{figure}[H]
        %\centering
        \includegraphics[width=15cm]{./fig/chapter4/inspect_2/manual/manual_1.pdf}
        \caption{検証2マニュアル1}
        \label{検証2マニュアル1}
    \end{figure}

    \vspace{4cm}%図の位置を正しくする!
    \begin{figure}[H]
        %\centering
        \includegraphics[width=15cm]{./fig/chapter4/inspect_2/manual/manual_2.pdf}
        \caption{検証2マニュアル2}
        \label{検証2マニュアル2}
    \end{figure}

    \vspace{4cm}%図の位置を正しくする!
    \begin{figure}[H]
        %\centering
        \includegraphics[width=15cm]{./fig/chapter4/inspect_2/manual/manual_3.pdf}
        \caption{検証2マニュアル3}
        \label{検証2マニュアル3}
    \end{figure}



  \subsection{検証結果}
  \subsection{考察}


