\chapter{検証}
\label{chap:poordirection}


\section{検証1}

\subsection{検証背景}

\subsubsection{検証目的}

password認証方式の,新規登録・認証の面倒さを解決するために,第3章の提案手法で
password認証方式に変わる,公開鍵暗号方式によるssh認証を用いた,WEBサービス認証の提案を行った。\\
しかしながら,提案だけだと,新規登録・認証の面倒さを解決していることの根拠に乏しい。
よって,検証を行い,第3章の提案手法は"面倒さの軽減"に効果的に繋がっているかの確認をする。


\subsubsection{検証手段}

検証の手段としては,実際に,以下の2っの認証方式の登録・認証を被験者に体験してもらう。
%また,実験マニュアルを作成し,実験者は被験者に対して,同じように接するようにする。

\begin{itemize}
  \item パスワード方式認証(以後 "パスワード認証"と記述する)
  \item 公開鍵暗号暗号方式によるssh認証(以後 "鍵認証”と記述する)
\end{itemize}
その後,2っの観点から,"面倒さ"を数値化する。

1っ目の観点は「時間」である。
"面倒さ"をアカウント登録・認証にかかる時間と推測し,計測化する。
詳しい詳細については,検証環境のマニュアルに記述する。

%まず,パスワード 形式による登録・認証の時間計測をそれぞれ行う。
% 次に,公開鍵暗号方式によるSSH 形式登録・認証の時間計測をそれぞれ行う。
% 最後に,上記の形式による時間計測を比較する。

2っ目の観点は「アンケート」である。
アンケートには,点数で答える方式,文字で記入する欄 の2っがあり,点数で答える方式により"面倒さ”を数値化する。
アンケートの細かい内容は,4.1.3の検証画面に記述する。

また,アンケートには"面倒さ"を数値化する以外にも,以下の2っの意味を込める。
1っ目の意味は次の通りである。
アカウント登録・認証にかかる時間 を,"面倒さ"と予想して検証しているが,その予想を確かめる必要がある。
アンケートを取ることにより,「アカウント登録・認証にかかる時間」と,「アンケートによる面倒さ」が比例していることを確認することで,予想を確かめることができる。
また,被験者の状態も確認することで,面倒と感じるのが,検証自体に対しての面倒さと関係があるのかを確認する。
2っ目の意味は次のとおりである。
記入欄で,改善点や感じたことの意見をもらうことで,今後の研究に生きるようなアンケートをもらう。








\subsection{検証環境}
 \subsubsection{検証場所}
 第3章で記述したとおり,検証場所は学科のVMを用いているため,学科のネットワーク内(有線LAN,wifiアクセスポイント{ie-ryukyu})
 から,アクセスして検証を行う。


 \subsubsection{検証画面}
 

 \subsubsection{検証マニュアル}

\subsection{検証結果}
\subsection{考察}

\section{検証2}